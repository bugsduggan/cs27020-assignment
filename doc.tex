\documentclass[a4paper, twoside]{article}

\setlength{\oddsidemargin}{0in} \setlength{\evensidemargin}{0in}
\setlength{\textwidth}{6.2in} \setlength{\topmargin}{-0.3in}
\setlength{\textheight}{9.8in}

\title{CS27020 Assignment - Career Planner} \author{Tom Leaman (thl5)}

\begin{document} \maketitle \newpage

\section{Un-normalised Analysis}

\begin{center} \begin{tabular}{|l|l|} \hline Company name & varchar(128) \\
\hline Company information & varchar(1024) \\ \hline Role & varchar(64) \\
\hline Deadline & date \\ \hline Required strength 1 & varchar(64) \\ \hline
Required strength 2 & varchar(64) \\ \hline Required strength 3 & varchar(64) \\
\hline Required qualification 1 & varchar(64) \\ \hline Required qualification 2
& varchar(64) \\ \hline My strength 1 & varchar(64) \\ \hline My strength 2 &
varchar(64) \\ \hline My strength 3 & varchar(64) \\ \hline Evidence 1 &
varchar(64) \\ \hline Evidence 2 & varchar(64) \\ \hline Evidence 3 &
varchar(64) \\ \hline Satisfaction 1 & varchar(64) \\ \hline Satisfaction 2 &
varchar(64) \\ \hline Satisfaction 3 & varchar(64) \\ \hline Qualification 1 &
varchar(64) \\ \hline Qualification 2 & varchar(64) \\ \hline Qualification 3 &
varchar(64) \\ \hline Date 1 & date \\ \hline Date 2 & date \\ \hline Date 3 &
date \\ \hline Grade 1 & varchar(64) \\ \hline Grade 2 & varchar(64) \\ \hline
Grade 3 & varchar(64) \\ \hline Aim 1 & varchar(64) \\ \hline Aim 2 &
varchar(64) \\ \hline Aim 3 & varchar(64) \\ \hline Relevance 1 & varchar(64) \\
\hline Relevance 2 & varchar(64) \\ \hline Relevance 3 & varchar(64) \\ \hline
CV & varchar(128) \\ \hline Covering letter & varchar(128) \\ \hline Date sent &
date \\ \hline Response & boolean \\ \hline Interview date & date \\ \hline
Outcome & varchar(1024) \\ \hline Reflection & varchar(1024) \\ \hline
\end{tabular} \end{center}

I have simply specified a field for each box on the form which I assume will
contain data. This means that the table already conforms to 1NF (each field
contains only one piece of data).

I have specified all Varchar type attributes with a limit which is a power of 2
(I believe this not only provides a more logically consistent scheme but also
may improve efficiency and aid optimisation when using particular database
engines). I would, however, hope to check these limits against representative
example data before committing to these limits.

\subsection{Primary key}
The primary key will be a composite of the Company name, Role and Deadline. I
have chosen to include the Deadline as it may be possible to apply for the same
position with the same company on more than one occasion.

\newpage
\subsection{Functional dependencies}
I have identified the following functional dependencies:

\begin{center}
  \textbf{Company name} $\rightarrow$ \textbf{Company information}
\end{center}
Each company will have its own blurb.

\begin{center}
  \textbf{Qualification} $\rightarrow$ \textbf{Date, Grade}
\end{center}
Each qualification will have been achieved on a single date with a single grade.

\begin{center}
  \textbf{Company name, Role, Deadline} $\rightarrow$ \textbf{Company
    information, Deadline, Required Strengths, Required Qualifications,
    Strengths, Evidence, Satisfaction, Qualifications, Dates, Grades, Aims,
    Relevance, CV, Covering Letter, Date sent, Response, Interview date,
  Outcome, Reflection}
\end{center}
The composite primary key uniquely identifies the remaining information in each
application.

It is possible that fields such as Strengths and Qualifications be
functionally dependent on the Required strengths and Required qualifications. I
have decided not to specify this as a functional dependency in this design; if
the user is applying for the same role at the same company at a different time,
it is possible (or even likely) that they will want to specify different
Strengths and Qualifications that they have gained in the intervening time.

\section{2nd Normal Form}
The Company information is only dependent on the Company name (part of the
primary key). So a new table will be created containing the Company name (which
will be the primary key) and the Company information. The Company information
field can then be removed from the main table.

\section{3rd Normal Form}
The Date and Grade of the Qualification are transitively dependent on the
Company name, Role and Deadline so to bring the database to 3rd normal form, we
will create a new table with Qualification (which will be the primary key), Date
and Grade. We can then remove the Date and Grade fields from the main table.

\end{document}
